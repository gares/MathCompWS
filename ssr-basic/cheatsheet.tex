\documentclass[12pt,a3paper]{article}
\usepackage{multicol}
\usepackage{multirow}
\usepackage[left=1cm,right=1cm,top=1cm,bottom=1.5cm,landscape]{geometry}
\usepackage{color}
\usepackage{listings}
\usepackage{hyperref}
\usepackage{amsthm}
\usepackage{needspace}

\newtheorem{lem}{Lemma}
\parindent=0pt

\newcommand{\C}[1]{\mbox{\lstinline`#1`}}

\definecolor{dkred}{rgb}{0.5,0,0}
\definecolor{dkblue}{rgb}{0,0.1,0.5}
\definecolor{dkgrey}{rgb}{0.3,0.3,0.3}
\definecolor{lightblue}{rgb}{0,0.5,0.5}
\definecolor{dkgreen}{rgb}{0,0.4,0}
\definecolor{dk2green}{rgb}{0.4,0,0}
\definecolor{dkviolet}{rgb}{0.6,0,0.8}
\definecolor{dkpink}{rgb}{0.8,0,0.9}
\def\lstlanguagefiles{defManSSR.tex,greylang.tex,lstlang0.sty,lstlang1.sty,lstlang2.sty,lstlang3.sty}
\lstset{language=SSR}
\lstset{moredelim=[is][\color{red}]{|*}{*|}}
\lstset{moredelim=[is][\color{blue}]{/*}{*/}}
\lstset{moredelim=[is][\dashuline]{|+}{+|}}
\lstset{moredelim=[is][\color{dkgrey}]{|.}{.|}}



\newcommand{\somethingfollows}{\needspace{2\baselineskip}}

\newcommand{\ssrc}[2]{\somethingfollows\C{#1}
	
\setlength{\leftskip}{1.5em}

#2

\setlength{\leftskip}{0cm}\vspace{0.4em}}
%\newcommand{\ssrc}[2]{\needspace{2\baselineskip}\C{#1}\vspace{0.4em}\\\hspace*{1.5em}\begin{minipage}{0.28\textwidth}#2 \hspace*{\fill}\end{minipage}\vspace{0.4em}}
\newcommand{\caveat}[1]{\fbox{caveat}: #1}
\begin{document}
\pagestyle{empty}

\begin{multicols*}{2}[\begin{center}\section*{\underbar{Basic Cheat Sheet}}\end{center}]

%%%%%%%%%%%%%%%%%%%%%%%%%%%%%%%%%%%%%%%%%%%%%%%%
\subsection*{Rewriting}
\ssrc{rewrite Eab (Exc b).}{
  Rewrite with \C{Eab} left to right, then with \C{Exc} by
  instantiating the first argument with \C{b}}
  \begin{tabular}{c@{$\to$}c} 
  \begin{minipage}{0.15\textwidth}\begin{lstlisting}
   Eab : a = b
   Exc : forall x, x = c
  =========
   P a
  \end{lstlisting}\end{minipage}
  &
  \begin{minipage}{0.15\textwidth}\begin{lstlisting}
   Eab : a = b
   Exc : forall x, x = c
  =========
   P c
  \end{lstlisting}\end{minipage}
  \end{tabular}
\ssrc{rewrite -Eab \{\}Eac.}{
Rewrite with \C{Eab} right to left then with \C{Eac}
left to right, finally clear \C{Eac}}
  \begin{tabular}{c@{$\to$}c} 
  \begin{minipage}{0.15\textwidth}\begin{lstlisting}
   Eab : a = b
   Eac : a = c
  =========
   P b
  \end{lstlisting}\end{minipage}
  &
  \begin{minipage}{0.15\textwidth}\begin{lstlisting}
   Eab : a = b
  =========
   P c
  \end{lstlisting}\end{minipage}
  \end{tabular}
\ssrc{rewrite !addnA.}{
Rewrite with \C{addnA}, associativity of addition, as many times as possible.}
  \begin{tabular}{c@{$\to$}c} 
  \begin{minipage}{0.15\textwidth}\begin{lstlisting}
  =========
  a + (b + (c + d))
  \end{lstlisting}\end{minipage}
  &
  \begin{minipage}{0.15\textwidth}\begin{lstlisting}
  =========
  a + b + c + d
  \end{lstlisting}\end{minipage}
  \end{tabular}

%%%%%%%%%%%%%%%%%%%%%%%%%%%%%%%%%%%%%%%%%%
\subsection*{Reasoning by cases or by induction}

\ssrc{case: n => [|p].}{Reson by cases on {\tt n}, name {\tt p} the predecessor}
  \begin{tabular}{c@{$\to$}cc} 
  \begin{minipage}{0.10\textwidth}\begin{lstlisting}
   n : nat
  =========
   P n 
  \end{lstlisting}\end{minipage}
  &
  \begin{minipage}{0.08\textwidth}\begin{lstlisting}
  =========
   P 0
  \end{lstlisting}\end{minipage}
  &
  \begin{minipage}{0.15\textwidth}\begin{lstlisting}
   p : nat
  =========
   P p.+1
  \end{lstlisting}\end{minipage}
  \end{tabular}

\ssrc{elim: n => [|m IHm].}{Perform an induction on {\tt n}}
  \begin{tabular}{c@{$\to$}cc} 
  \begin{minipage}{0.10\textwidth}\begin{lstlisting}
   n : nat
  =========
   P n 
  \end{lstlisting}\end{minipage}
  &
  \begin{minipage}{0.08\textwidth}\begin{lstlisting}
  =========
   P 0
  \end{lstlisting}\end{minipage}
  &
  \begin{minipage}{0.15\textwidth}\begin{lstlisting}
   m : nat
   IHm : P m
  =========
   P m.+1
  \end{lstlisting}\end{minipage}
  \end{tabular}

  \ssrc{elim: s => // x xs IHxs.}{Get rid of trivial goals,
  hence no {\tt [ .. | .. ]}}
  \begin{tabular}{c@{$\to$}c} 
  \begin{minipage}{0.10\textwidth}\begin{lstlisting}
   s : seq nat
  =========
   P s 
  \end{lstlisting}\end{minipage}
  &
  \begin{minipage}{0.15\textwidth}\begin{lstlisting}
   x : nat
   xs : seq nat
   IHxs : P xs
  =========
   P (x :: xs)
  \end{lstlisting}\end{minipage}
  \end{tabular}

\columnbreak

%%%%%%%%%%%%%%%%%%%%%%%%%%%%%%%%%%%%%%%%%%%%%

\subsection*{Naming and processing assumptions}

\ssrc{move=> x /lemma px}{
  Name the first item \C{x} then
  view the top item via \C{lemma} and name the result \C{qx}.
  \C{lemma} has type \C{forall a, P a -> Q a}, or \C{reflect P Q}}.
  \begin{tabular}{c@{$\to$}c} 
  \begin{minipage}{0.15\textwidth}\begin{lstlisting}
  =========
   forall x,
    P x -> R x -> G
  \end{lstlisting}\end{minipage}
  &
  \begin{minipage}{0.15\textwidth}\begin{lstlisting}
   x : T
   qx : Q x
  =========
   R x -> G
  \end{lstlisting}\end{minipage}
  \end{tabular}

\ssrc{move=> /andP[/eqP-> pb]}{
  Process the top item with the view \C{andP}, then destruct the
  resulting conjunction, use \C{eqP} on the first item and then
  rewrite with it, finally name the rest \C{pb}.}
  \begin{tabular}{c@{$\to$}c} 
  \begin{minipage}{0.18\textwidth}\begin{lstlisting}
  a, b : nat
  =========
  (a == 7) && 10 <= b -> a + 3 <= b
  \end{lstlisting}\end{minipage}
  &
  \begin{minipage}{0.15\textwidth}\begin{lstlisting}
  a, b : nat
  pb : 10 <= b
  =========
   7 + 3 <= b
  \end{lstlisting}\end{minipage}
  \end{tabular}

\ssrc{move=> /= \{pa\}}{
  Simplify the goal, then clear \C{pa} from the context}

  \begin{tabular}{c@{$\to$}c} 
  \begin{minipage}{0.18\textwidth}\begin{lstlisting}
  a : nat
  pa : a != 3
  =========
  (3 == 7) || (10 <= a)
  \end{lstlisting}\end{minipage}
  &
  \begin{minipage}{0.15\textwidth}\begin{lstlisting}
  a : nat
  =========
  (10 <= a)
  \end{lstlisting}\end{minipage}
  \end{tabular}

%%%%%%%%%%%%%%%%%%%%%%%%%%%%%%%%%%%%%%%%%%%%%%%%%%%%
\subsection*{Back and Forward chaining}

\ssrc{apply: H.}{
  Apply \C{H} to the current goal}
  \begin{tabular}{c@{$\to$}c} 
  \begin{minipage}{0.15\textwidth}\begin{lstlisting}
   H : A -> B
  =========
   B
  \end{lstlisting}\end{minipage}
  &
  \begin{minipage}{0.15\textwidth}\begin{lstlisting}
  =========
   A
  \end{lstlisting}\end{minipage}
  \end{tabular}

\ssrc{apply/subsetP.}{
  Apply the view \C{subsetP} to the current goal}
  \begin{tabular}{c@{$\to$}c} 
  \begin{minipage}{0.13\textwidth}\begin{lstlisting}
   A, B : {set T}
  =========
   B \subset A
  \end{lstlisting}\end{minipage}
  &
  \begin{minipage}{0.17\textwidth}\begin{lstlisting}
   A, B : {set T}
  =========
   forall x, x \in B -> x \in A
  \end{lstlisting}\end{minipage}
  \end{tabular}

\ssrc{have pa : P a.}{
  Open a new goal for \C{P a}. Once resolved
  introduce a new entry in the context for it named \C{pa}}
  \begin{tabular}{c@{$\to$}cc} 
  \begin{minipage}{0.10\textwidth}\begin{lstlisting}
   a : T
  =========
   G
  \end{lstlisting}\end{minipage}
  &
  \begin{minipage}{0.10\textwidth}\begin{lstlisting}
   a : T
  =========
   P a
  \end{lstlisting}\end{minipage}
  &
  \begin{minipage}{0.10\textwidth}\begin{lstlisting}
   a : T
   pa : P a
  =========
   G
  \end{lstlisting}\end{minipage}
  \end{tabular}

\ssrc{by [].}{
  Prove the goal by trivial means, or fail}
  \begin{tabular}{c@{$\to$}c} 
  \begin{minipage}{0.15\textwidth}\begin{lstlisting}
  =========
   0 <= n
  \end{lstlisting}\end{minipage}
  &
  \begin{minipage}{0.15\textwidth}\begin{lstlisting}
  \end{lstlisting}\end{minipage}
  \end{tabular}


\end{multicols*}

\end{document}

